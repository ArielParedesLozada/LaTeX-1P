\documentclass[runningheads]{llncs}
\usepackage[spanish]{babel}
\usepackage{graphicx}

\begin{document}
\title{GitLab}
\author{David Barragán\inst{1} \and
        Joel Bonilla\inst{1} \and
        Josué García\inst{1} \and
        David Manjarrez\inst{1} \and
        Ariel Paredes Lozada\inst{1}
        }
\institute{Universidad Técnica de Ambato, Ambato, Ecuador
\email{secretariageneral@uta.edu.ec}}
\date{16 de octubre de 2024}

\maketitle
\begin{abstract}
        %Abstract de 120 palabras
        El trabajo es sobre GitLab
\end{abstract}
\keywords{GitLab \and VCS \and características}
\section{Introducción}
%Qué es el sistema de control de versiones y así
%Panorama para usar GitLab
jsdbckasjdbncjkasdkljclsdc\cite{safari2020analysis}
\section{Metodología}
%Qué artículos se investigarons
%Bajo qué criterios se escogieron
%De dónde se lo sacó
\subsection{Preguntas de investigación}
%Pon en taxonomía de Bloom
%¿Qué es GitLab?->Objetivo: Identificar qué es GitLab y sus características importantes
%¿Cuáles son las funciones de GitLab?->Objetivo: Qué hace GitLab, sus casos de uso y ejemplos de cosas que usan GitLab
%¿Cuál es la historia de GitLab?->Objetivo: Poner en perspectiva a GitLab en la historia y frente a otros VCSs
%¿Cuáles son las ventajas y desventajas de GitLab?-> Objetivo: Sopesar los pro y contras de GitLab para determinar su uso en proyectos
\subsection{Exploración de documentos}
%Se hizo una búsqueda en tal y tal sitios web.
%Se sacaron los artículostal y cual de Google Scholar, los demás de este otro sitio
\subsection{Selección de obras}
%Criterios de selección
%Obras desde el 2020, con relación a GitLab
\subsection{Adquisición de datos significativos}
%Resumen breve de las obras, características importantes
\section{Resultados}
\subsection{GitLab: Concepto y características}
%Eso po
\subsection{Funciones de GitLab}
%Funciones de GitLab
\subsection{Historia de GitLab}
%Historia de GitLab
%Asegurarse de mencionar a BitBucket y otros sistemas similares
\subsection{Ventajas y desventajas}
%Ventajas y deventajas de GitLab
%Tabla de comparación con otros sistemas similares
\section{Discusión}
%Contextualizar la investigación
%Problemas con la invetigación. Recomendaciones 
\section{Conclusiones}
%Resumen de GitLab, qué hace, para qué se lo usa y qué beneficios tiene
\bibliographystyle{plain}
\bibliography{ref}
\end{document}