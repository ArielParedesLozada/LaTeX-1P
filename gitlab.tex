\documentclass[runningheads]{llncs}
\usepackage[spanish]{babel}
\usepackage{graphicx}
\usepackage{tabularx}

\begin{document}
\title{GitLab}
\author{David Barragán\inst{1} \and
        Joel Bonilla\inst{1} \and
        Josué García\inst{1} \and
        David Manjarrez\inst{1} \and
        Ariel Paredes Lozada\inst{1}
        }
\institute{Universidad Técnica de Ambato, Ambato, Ecuador\\
\email{secretariageneral@uta.edu.ec}}

\maketitle
\begin{abstract}
        %Abstract de 120 palabras
        El control de versiones, el testeo y monitoreo son una partes fundamentales dentro del desarrollo del software.
        En la actualidad, existen múltiples herramientas que automatizan este proceso, que se engloban bajo el nombre de DevOps.
        Una de estas herramientas de DevOps es GitLab, que facilita el ciclo de desarrollo y facilita la colaboración.
        Es necesario conocer las características, funciones, ventajas y desventajas de esta herramienta con el fin de 
        sopesar su uso y determinar la viabilidad de su uso en proyectos. Por este mismo motivo, es enecesario conocer la
        historia de GitLab, y qué papel juega dentro del panorama de las demás herramientas de DevOps.
        Con este fin, se ha hecho una investigación para determinar las características más importantes de GitLab y
        sintetizarlo en un informe. 
\end{abstract}
\keywords{GitLab \and DevOps \and funciones}
\section{Introducción}
%Qué es el sistema de control de versiones y así
%Panorama para usar GitLab
El desarrollo de software es una tarea colaborativa, con múltiples desarrolladores trabajando juntos en
un mismo proyecto. Sin embargo, manejar un proyecto así es difícil \cite{fairbanks2023analyzing}. Existen
muchos problemas que pueden aparecer en desarrollo, relacionados al manejo de los cambios por parte de los
desarrolladores, a las pruebas continuas en el código y a la integración con las distintas partes.\\
Con el fin de resolver estos problemas, se han propuesto el uso de herramientas que faciliten la colaboración
y monitoreo dentro del desarrollo de software. Este conjunto de herramientas y prácticas que buscan facilitar
y automatizar los procesos de desarrollo de software se denominan DevOps \cite{gitlab2022gitlab}.\\
Una de las herramientas de DevOps más conocidas es GitLab, que permite automatizar el ciclo de desarrollo,
desde la planificación y creación hasta las pruebas y el monitoreo de aplicaciones. GitLab es una plataforma
de gestión de repositorios que permite la colaboración entre desarrolladores, así como la automatización de
pruebas y el despliegue del software \cite{safari2020analysis}. GitLab es similar a GitHub, en el sentido de que
ofrece la gestión de repositorios y permite el control de versiones por distintas partes, pero difiere en
que provee de más servicios, enfocados en la integración y entrega continuas (CI/CD, por sus siglas en inglés) \cite{safari2020analysis}.\\
Con el fin de saber más acerca de GitLab y sus aplicaciones, se ha realizado una investigación en
distintas fuentes. Se busca entender las características, ventajas y desventajas de GitLab, con el fin
de explicar las motivaciones de su uso dentro de un proyecto de software. Así mismo, se busa reconocer el
lugar en donde se encuentra GitLab dentro de las herramientas DevOps, mediante una comparación frente a las demás
opciones existentes en el mercado.
\section{Metodología}
%Qué artículos se investigaron
%Bajo qué criterios se escogieron
%De dónde se lo sacó
Se investigó una serie de artículos relacionados al uso de GitLab en distintos proyectos, con el fin
de analizar su uso en el desarrollo de software. En total, se investigaron cinco artículos de fuentes confiables.
Entre estas fuentes se encuentran: Google Scholar y la página web Academia.\\
Los criterios de búsqueda que se usaron para los artículos fueron:
\begin{itemize}
        \item Tener los términos clave "GitLab" y "development", "characteristics" o "DevOps"
        \item Estar publicado desde el 2020 hasta la actualidad
        \item Estar publicado por una institución oficial
        \item De preferencia, que sea una investigación publicada en inglés
\end{itemize}
Se obtuvieron ocho artículos bajo este enfoque. Con el fin de extraer la mayor cantidad de datos de
esta investigación, se siguen la metodología de Revisión Sitemática de la Literatura, con un enfoque
cualitativo.
\subsection{Preguntas de investigación}
%Pon en taxonomía de Bloom
%¿Qué es GitLab?->Objetivo: Identificar qué es GitLab y sus características importantes
%¿Cuáles son las funciones de GitLab?->Objetivo: Qué hace GitLab, sus casos de uso y ejemplos de cosas que usan GitLab
%¿Cuál es la historia de GitLab?->Objetivo: Poner en perspectiva a GitLab en la historia y frente a otros VCSs
%¿Cuáles son las ventajas y desventajas de GitLab?-> Objetivo: Sopesar los pro y contras de GitLab para determinar su uso en proyectos
Se buscan responder cuatro interrogantes con la investigación:
\begin{itemize}
        \item \textbf{RQ1}: ¿Qué es GitLab?\\
        \textit{RQ1-Objetivo}: Identificar las características principales de GitLab, como tipo de licencia y curva de aprendizaje
        \item \textbf{RQ2}: ¿Cuáles son las funciones de GitLab?\\
        \textit{RQ2-Objetivo}: Describir las principales funciones, casos de uso y ejemplos de aplicaciones hechas con GitLab
        \item \textbf{RQ3}: ¿Cuál es la historia de GitLab?\\
        \textit{RQ3-Objetivo}: Compilar el desarrollo de GitLab a lo largo del tiempo y contrastarlo con otras herramientas similares
        \item \textbf{RQ4}: ¿Cuáles son las ventajas y desventajas de GitLab?\\
        \textit{RQ4-Objetivo}: Examinar los pros y contras del desarrollo con GitLab frente a otras herramientas DevOps y enfoques de desarrollo
\end{itemize}
\subsection{Exploración de documentos}
%Se hizo una búsqueda en tal y tal sitios web.
%Se sacaron los artículostal y cual de Google Scholar, los demás de este otro sitio
Se hizo una búsqueda en los sitios web de Google Scholar, Academia, IEEE Xplore, Research Gate e incluso PubMed Central.
Se encontraron los siguientes artículos.
%PD: Pon más artículos si encuentras
\begin{table}[ht!]
        \centering
        \begin{tabular}{l | c | l}
                Base de Datos & Número de artículos consultados & URL \\
                \hline
                \hline
                IEEE Xplorer & 3 & https://ieeexplore.ieee.org/ \\
                Research Gate & 3 & https://www.researchgate.net/ \\
                PubMed & 1 & https://pubmed.ncbi.nlm.nih.gov/ \\
                Google Scholar & 1 & https://scholar.google.com/ 
        \end{tabular}
        \caption{Bases de datos consultadas}
        \label{table:1}
\end{table}
\subsection{Selección de obras}
%Criterios de selección
%Obras desde el 2020, con relación a GitLab
Se encontraron una serie de artículos que hacen referencia a GitLab y a herramientas DevOps. En total se
recuperaron 5 artículos que hacen directa referencia a GitLab, y que proveen información acerca de otras
herramientas de desarrollo.\\
Los artículos se basaron en los siguientes criterios:
\begin{itemize}
        \item Tener los términos clave "GitLab", "development", "characteristics" o "DevOps"
        \item Estar publicado desde el 2020 hasta la actualidad
        \item Estar publicado por una institución oficial
        \item De preferencia, que sea una investigación publicada en inglés
\end{itemize}
Todos los artículos encontrados caen dentro de estos criterios de selección.
%\subsection{Adquisición de datos significativos}
%Resumen breve de las obras, características importantes
\section{Resultados}
\subsection{RQ1: ¿Qué es GitLab?}
%Eso po
GitLab es una herramienta web para DevOps \cite{gitlab2022gitlab}. Es una plataforma web que, además de ofrecer
servicios de gestión de repositorios, provee servicios de integración y entrega continua (CI/CD) \cite{fairbanks2023analyzing}.
GitLab da alojamiento a código, de la misma manera que lo hacen otras plataformas como GitHub, BitBucket, Source-Forge
y LaunchPad \cite{safari2020structural}. Sin embargo, GitLab se diferencia de estas principalmente en que GitLab ofrece más
herramientas para el ciclo de vida del software, incluyendo la ya mencionada CI/CD, la gestión de proyectos e incluye
herramientas como wikis de proyecto y seguimiento de problemas, lo que facilita el desarrollo y el manejo de la gestión
en gran medida \cite{safari2020structural}. Además, GitLab tiene un enfoque en la privacidad y el control del código, ya que
permite a las organizaciones autoalojar sus recursos y no depender de servidores externos. También se debe tomar en cuenta que
GitLab permite la colaboración no sólo entre desarrolladores, sino también con otros roles, como diseñadores y
gerentes de proyecto \cite{choudhury2020gitlab}. Esto permite una mejor colaboración entre las partes relacionadas al
desarrollo de un proyecto.\\
Otra de las características fundamentales de GitLab es que es un proyecto de código abierto, lo que permite a los usuarios
modificar el código fuente de la plataforma. Esto ofrece una gran flexibilidad a la hora de personalizarlo, lo que contrasta
con plataformas similares, que son propiedad de una empresa, como es el caso de GitHub \cite{safari2020structural}. También es
necesario tener en cuenta que GitLab, aunque se pensó inicialmente como una plataforma enfocada a la gestion y manejo del software,
puede ser usada para otra gran cantidad de proyectos. El modelo de trabajo de GitLab, enfocado en la colaboración remota y asincrónica,
la transparencia y la gestión de procesos, permite que muchas otras empresas puedan usar GitLab para implementar modelos de
trabajo remoto \cite{choudhury2020gitlab}.\\
GitLab tiene las siguientes características más relevantes:
\begin{enumerate}
        \item \textit{Año de creación}: 2011
        \item \textit{Licencia}: MIT para la versión gratis, licencia propietaria para la pagada
        \item \textit{Empresa propietaria}: GitLab Inc.
        \item \textit{Lenguaje de implementación}: Ruby, con Ruby on Rails
\end{enumerate}
Existe una gran cantidad de proyectos que usan GitLab, pertenecientes no sólo a empresas propietarias
sino que también lo usan proyectos independientes que integren a programadores de distintas zonas
geográficas \cite{safari2020structural}. Desde hace algún tiempo, varios equipos han comenzado a usar GitLab
gracias a sus capacidades en la integración y entrega continua.\\
Sin embargo, GitLab también tiene limitaciones en comparación a plataformas más especializadas, como lo es GitHub
Por ejemplo, GitLab tienen menos capacidad a la hora de visualiazr el código, en contraste con GitHub, por ejemplo.
Además, algunas de la comunidad de GitLab es mucho menor que la de otras plataformas, lo que puede dificultar su uso.
Otra característica importante de GitLab es que maneja un flujo de trabajo particular, con estándares propios referentes
a los nombres de la ramas y a cómo se deben unir entre sí \cite{devineni1version}. Este modelo de ramificación, llamado
GitLab Flow, está basado en GitHub flow, y describe 
%TODO: PON EL GITLAB Flow (REF-GITLAB)
\subsection{Funciones de GitLab}
%Funciones de GitLab
GitLab ofrece muchas funciones que facilitan el desarrollo de software y la colaboración entre equipos. Esto lo
logra mediante varias funciones, enfocadas en la integración y entrega continua, la colaboración entre desarrolladores,
el análisis de desempeño, entre otras, relacionadas a la filosofía de DevOps \cite{fairbanks2023analyzing}.\\
Entre las funciones más importantes que ofrece GitLab se encuentran:
\subsubsection{Control de versiones}
GitLab utiliza a Git como sistema de control de versiones \cite{choudhury2020gitlab}. Esto significa que usa un
manejo de versones basado en commits, ramas y todas los servicios que Git tiene. Esto también significa que GitLab
se basa en sistema de control de versiones distribuido, en el cual distintos miembros del equipo pueden trabajar
de forma independiente y luego unir sus cambios \cite{alvin2023devops}.
\begin{figure}[htbp]
        \centering
        \includegraphics[width=0.5\textwidth]{Distributed.png}
        \caption{Sistema distribuido}
        \label{fig:sys-dis}
\end{figure}
\subsubsection{Integración continua}
Esto hace referencia a la práctica de integrar y probar el código constantemente con el fin de evitar el surgiemiento
de bugs y errores en etapas tempranas. Esto se hace mediante un proceso automatizado de pruebas y construcción de artefactos
cada vez que se realiza un commit a las ramas de desarrollo \cite{alvin2023devops,uddin2023comparative}. Esto se logra
mediante un script especializado, que se ejecuta al momento de enviar cambios al repositorio remoto.\\
En el caso de GitLab, este script se lo guarda en el archivo \textsf{.gitlab-ci.yml}. Para las pruebas, se usa el software
de GitLab Runner, el cual se puede personalizar según las necesidades de los desarrolladores.
\subsubsection{Entrega continua}
Es una función referente a ñla 
\subsection{Historia de GitLab}
%Historia de GitLab
%Asegurarse de mencionar a BitBucket y otros sistemas similares
\begin{table}
        \centering
        \begin{tabularx}{1.1\textwidth}{X || X | X | X | X | X | X}
            Herramienta & Curva de aprendizaje & Tipo de licencia & Empresa propietaria & Año de creación & Lenguaje usado & Ejemplos de uso \\
            \hline
            \hline
            GitLab & Media & MIT License & GitLab Inc. & 2011 & Ruby (framework Ruby on Rails), Javascript, Go & Varios proyectos de múltiples campos
        \end{tabularx}
        \caption{Características de GitLab}
        \label{table:2}
\end{table}
\subsection{Ventajas y desventajas}
%Ventajas y deventajas de GitLab
%Tabla de comparación con otros sistemas similares
\section{Discusión}
%Contextualizar la investigación
%Problemas con la invetigación. Recomendaciones 
\section{Conclusiones}
%Resumen de GitLab, qué hace, para qué se lo usa y qué beneficios tiene
\bibliographystyle{plain}
\bibliography{ref}
\end{document}