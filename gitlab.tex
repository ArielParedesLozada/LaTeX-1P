\documentclass[runningheads]{llncs}
\usepackage[spanish]{babel}
\usepackage{graphicx}

\begin{document}
\title{GitLab}
\author{David Barragán\inst{1} \and
        Joel Bonilla\inst{1} \and
        Josué García\inst{1} \and
        David Manjarrez\inst{1} \and
        Ariel Paredes Lozada\inst{1}
        }
\institute{Universidad Técnica de Ambato, Ambato, Ecuador\\
\email{secretariageneral@uta.edu.ec}}

\maketitle
\begin{abstract}
        %Abstract de 120 palabras
        El control de versiones, el testeo y monitoreo son una partes fundamentales dentro del desarrollo del software.
        En la actualidad, existen múltiples herramientas que automatizan este proceso, que se engloban bajo el nombre de DevOps.
        Una de estas herramientas de DevOps es GitLab, que facilita el ciclo de desarrollo y facilita la colaboración.
        Es necesario conocer las características, funciones, ventajas y desventajas de esta herramienta con el fin de 
        sopesar su uso y determinar la viabilidad de su uso en proyectos. Por este mismo motivo, es enecesario conocer la
        historia de GitLab, y qué papel juega dentro del panorama de las demás herramientas de DevOps.
        Con este fin, se ha hecho una investigación para determinar las características más importantes de GitLab y
        sintetizarlo en un informe. 
\end{abstract}
\keywords{GitLab \and DevOps \and funciones}
\section{Introducción}
%Qué es el sistema de control de versiones y así
%Panorama para usar GitLab
El desarrollo de software es una tarea colaborativa, con múltiples desarrolladores trabajando juntos en
un mismo proyecto. Sin embargo, manejar un proyecto así es difícil \cite{fairbanks2023analyzing}. Existen
muchos problemas que pueden aparecer en desarrollo, relacionados al manejo de los cambios por parte de los
desarrolladores, a las pruebas continuas en el código y a la integración con las distintas partes.\\
Con el fin de resolver estos problemas, se han propuesto el uso de herramientas que faciliten la colaboración
y monitoreo dentro del desarrollo de software. Este conjunto de herramientas y prácticas que buscan facilitar
y automatizar los procesos de desarrollo de software se denominan DevOps \cite{gitlab2022gitlab}.\\
Una de las herramientas de DevOps más conocidas es GitLab, que permite automatizar el ciclo de desarrollo,
desde la planificación y creación hasta las pruebas y el monitoreo de aplicaciones. GitLab es una plataforma
de gestión de repositorios que permite la colaboración entre desarrolladores, así como la automatización de
pruebas y el despliegue del software \cite{safari2020analysis}. GitLab es similar a GitHub, en el sentido de que
ofrece la gestión de repositorios y permite el control de versiones por distintas partes, pero difiere en
que provee de más servicios, enfocados en la integración y entrega continuas (CI/CD, por sus siglas en inglés) \cite{safari2020analysis}.\\
Con el fin de saber más acerca de GitLab y sus aplicaciones, se ha realizado una investigación en
distintas fuentes. Se busca entender las características, ventajas y desventajas de GitLab, con el fin
de explicar las motivaciones de su uso dentro de un proyecto de software. Así mismo, se busa reconocer el
lugar en donde se ecnuentra GitLab dentro de las herramientas DevOps, mediante una comparación frente a las demás
opciones existentes en el mercado.
\section{Metodología}
%Qué artículos se investigaron
%Bajo qué criterios se escogieron
%De dónde se lo sacó
Se investigó una serie de artículos relacionados al uso de GitLab en distintos proyectos, con el fin
de analizar su uso en el desarrollo de software. En total, se investigaron cinco artículos de fuentes confiables.
Entre estas fuentes se encuentran: Google Scholar y la página web Academia.\\
Los criterios de búsqueda que se usaron para los artículos fueron:
\begin{itemize}
        \item Tener los términos clave "GitLab", "development", "characteristics" o "DevOps"
        \item Estar publicado desde el 2020 hasta la actualidad
        \item Estar publicado por una institución oficial
\end{itemize}
Se obtuvieron cinco artículos bajo este enfoque.
\subsection{Preguntas de investigación}
%Pon en taxonomía de Bloom
%¿Qué es GitLab?->Objetivo: Identificar qué es GitLab y sus características importantes
%¿Cuáles son las funciones de GitLab?->Objetivo: Qué hace GitLab, sus casos de uso y ejemplos de cosas que usan GitLab
%¿Cuál es la historia de GitLab?->Objetivo: Poner en perspectiva a GitLab en la historia y frente a otros VCSs
%¿Cuáles son las ventajas y desventajas de GitLab?-> Objetivo: Sopesar los pro y contras de GitLab para determinar su uso en proyectos
\subsection{Exploración de documentos}
%Se hizo una búsqueda en tal y tal sitios web.
%Se sacaron los artículostal y cual de Google Scholar, los demás de este otro sitio
\subsection{Selección de obras}
%Criterios de selección
%Obras desde el 2020, con relación a GitLab
\subsection{Adquisición de datos significativos}
%Resumen breve de las obras, características importantes
\section{Resultados}
\subsection{GitLab: Concepto y características}
%Eso po
\subsection{Funciones de GitLab}
%Funciones de GitLab
\subsection{Historia de GitLab}
%Historia de GitLab
%Asegurarse de mencionar a BitBucket y otros sistemas similares
\subsection{Ventajas y desventajas}
%Ventajas y deventajas de GitLab
%Tabla de comparación con otros sistemas similares
\section{Discusión}
%Contextualizar la investigación
%Problemas con la invetigación. Recomendaciones 
\section{Conclusiones}
%Resumen de GitLab, qué hace, para qué se lo usa y qué beneficios tiene
\bibliographystyle{plain}
\bibliography{ref}
\end{document}